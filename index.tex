% Options for packages loaded elsewhere
\PassOptionsToPackage{unicode}{hyperref}
\PassOptionsToPackage{hyphens}{url}
\PassOptionsToPackage{dvipsnames,svgnames,x11names}{xcolor}
%
\documentclass[
  letterpaper,
  DIV=11,
  numbers=noendperiod]{scrreprt}

\usepackage{amsmath,amssymb}
\usepackage{lmodern}
\usepackage{iftex}
\ifPDFTeX
  \usepackage[T1]{fontenc}
  \usepackage[utf8]{inputenc}
  \usepackage{textcomp} % provide euro and other symbols
\else % if luatex or xetex
  \usepackage{unicode-math}
  \defaultfontfeatures{Scale=MatchLowercase}
  \defaultfontfeatures[\rmfamily]{Ligatures=TeX,Scale=1}
\fi
% Use upquote if available, for straight quotes in verbatim environments
\IfFileExists{upquote.sty}{\usepackage{upquote}}{}
\IfFileExists{microtype.sty}{% use microtype if available
  \usepackage[]{microtype}
  \UseMicrotypeSet[protrusion]{basicmath} % disable protrusion for tt fonts
}{}
\makeatletter
\@ifundefined{KOMAClassName}{% if non-KOMA class
  \IfFileExists{parskip.sty}{%
    \usepackage{parskip}
  }{% else
    \setlength{\parindent}{0pt}
    \setlength{\parskip}{6pt plus 2pt minus 1pt}}
}{% if KOMA class
  \KOMAoptions{parskip=half}}
\makeatother
\usepackage{xcolor}
\setlength{\emergencystretch}{3em} % prevent overfull lines
\setcounter{secnumdepth}{5}
% Make \paragraph and \subparagraph free-standing
\ifx\paragraph\undefined\else
  \let\oldparagraph\paragraph
  \renewcommand{\paragraph}[1]{\oldparagraph{#1}\mbox{}}
\fi
\ifx\subparagraph\undefined\else
  \let\oldsubparagraph\subparagraph
  \renewcommand{\subparagraph}[1]{\oldsubparagraph{#1}\mbox{}}
\fi

\usepackage{color}
\usepackage{fancyvrb}
\newcommand{\VerbBar}{|}
\newcommand{\VERB}{\Verb[commandchars=\\\{\}]}
\DefineVerbatimEnvironment{Highlighting}{Verbatim}{commandchars=\\\{\}}
% Add ',fontsize=\small' for more characters per line
\usepackage{framed}
\definecolor{shadecolor}{RGB}{241,243,245}
\newenvironment{Shaded}{\begin{snugshade}}{\end{snugshade}}
\newcommand{\AlertTok}[1]{\textcolor[rgb]{0.68,0.00,0.00}{#1}}
\newcommand{\AnnotationTok}[1]{\textcolor[rgb]{0.37,0.37,0.37}{#1}}
\newcommand{\AttributeTok}[1]{\textcolor[rgb]{0.40,0.45,0.13}{#1}}
\newcommand{\BaseNTok}[1]{\textcolor[rgb]{0.68,0.00,0.00}{#1}}
\newcommand{\BuiltInTok}[1]{\textcolor[rgb]{0.00,0.23,0.31}{#1}}
\newcommand{\CharTok}[1]{\textcolor[rgb]{0.13,0.47,0.30}{#1}}
\newcommand{\CommentTok}[1]{\textcolor[rgb]{0.37,0.37,0.37}{#1}}
\newcommand{\CommentVarTok}[1]{\textcolor[rgb]{0.37,0.37,0.37}{\textit{#1}}}
\newcommand{\ConstantTok}[1]{\textcolor[rgb]{0.56,0.35,0.01}{#1}}
\newcommand{\ControlFlowTok}[1]{\textcolor[rgb]{0.00,0.23,0.31}{#1}}
\newcommand{\DataTypeTok}[1]{\textcolor[rgb]{0.68,0.00,0.00}{#1}}
\newcommand{\DecValTok}[1]{\textcolor[rgb]{0.68,0.00,0.00}{#1}}
\newcommand{\DocumentationTok}[1]{\textcolor[rgb]{0.37,0.37,0.37}{\textit{#1}}}
\newcommand{\ErrorTok}[1]{\textcolor[rgb]{0.68,0.00,0.00}{#1}}
\newcommand{\ExtensionTok}[1]{\textcolor[rgb]{0.00,0.23,0.31}{#1}}
\newcommand{\FloatTok}[1]{\textcolor[rgb]{0.68,0.00,0.00}{#1}}
\newcommand{\FunctionTok}[1]{\textcolor[rgb]{0.28,0.35,0.67}{#1}}
\newcommand{\ImportTok}[1]{\textcolor[rgb]{0.00,0.46,0.62}{#1}}
\newcommand{\InformationTok}[1]{\textcolor[rgb]{0.37,0.37,0.37}{#1}}
\newcommand{\KeywordTok}[1]{\textcolor[rgb]{0.00,0.23,0.31}{#1}}
\newcommand{\NormalTok}[1]{\textcolor[rgb]{0.00,0.23,0.31}{#1}}
\newcommand{\OperatorTok}[1]{\textcolor[rgb]{0.37,0.37,0.37}{#1}}
\newcommand{\OtherTok}[1]{\textcolor[rgb]{0.00,0.23,0.31}{#1}}
\newcommand{\PreprocessorTok}[1]{\textcolor[rgb]{0.68,0.00,0.00}{#1}}
\newcommand{\RegionMarkerTok}[1]{\textcolor[rgb]{0.00,0.23,0.31}{#1}}
\newcommand{\SpecialCharTok}[1]{\textcolor[rgb]{0.37,0.37,0.37}{#1}}
\newcommand{\SpecialStringTok}[1]{\textcolor[rgb]{0.13,0.47,0.30}{#1}}
\newcommand{\StringTok}[1]{\textcolor[rgb]{0.13,0.47,0.30}{#1}}
\newcommand{\VariableTok}[1]{\textcolor[rgb]{0.07,0.07,0.07}{#1}}
\newcommand{\VerbatimStringTok}[1]{\textcolor[rgb]{0.13,0.47,0.30}{#1}}
\newcommand{\WarningTok}[1]{\textcolor[rgb]{0.37,0.37,0.37}{\textit{#1}}}

\providecommand{\tightlist}{%
  \setlength{\itemsep}{0pt}\setlength{\parskip}{0pt}}\usepackage{longtable,booktabs,array}
\usepackage{calc} % for calculating minipage widths
% Correct order of tables after \paragraph or \subparagraph
\usepackage{etoolbox}
\makeatletter
\patchcmd\longtable{\par}{\if@noskipsec\mbox{}\fi\par}{}{}
\makeatother
% Allow footnotes in longtable head/foot
\IfFileExists{footnotehyper.sty}{\usepackage{footnotehyper}}{\usepackage{footnote}}
\makesavenoteenv{longtable}
\usepackage{graphicx}
\makeatletter
\def\maxwidth{\ifdim\Gin@nat@width>\linewidth\linewidth\else\Gin@nat@width\fi}
\def\maxheight{\ifdim\Gin@nat@height>\textheight\textheight\else\Gin@nat@height\fi}
\makeatother
% Scale images if necessary, so that they will not overflow the page
% margins by default, and it is still possible to overwrite the defaults
% using explicit options in \includegraphics[width, height, ...]{}
\setkeys{Gin}{width=\maxwidth,height=\maxheight,keepaspectratio}
% Set default figure placement to htbp
\makeatletter
\def\fps@figure{htbp}
\makeatother
\newlength{\cslhangindent}
\setlength{\cslhangindent}{1.5em}
\newlength{\csllabelwidth}
\setlength{\csllabelwidth}{3em}
\newlength{\cslentryspacingunit} % times entry-spacing
\setlength{\cslentryspacingunit}{\parskip}
\newenvironment{CSLReferences}[2] % #1 hanging-ident, #2 entry spacing
 {% don't indent paragraphs
  \setlength{\parindent}{0pt}
  % turn on hanging indent if param 1 is 1
  \ifodd #1
  \let\oldpar\par
  \def\par{\hangindent=\cslhangindent\oldpar}
  \fi
  % set entry spacing
  \setlength{\parskip}{#2\cslentryspacingunit}
 }%
 {}
\usepackage{calc}
\newcommand{\CSLBlock}[1]{#1\hfill\break}
\newcommand{\CSLLeftMargin}[1]{\parbox[t]{\csllabelwidth}{#1}}
\newcommand{\CSLRightInline}[1]{\parbox[t]{\linewidth - \csllabelwidth}{#1}\break}
\newcommand{\CSLIndent}[1]{\hspace{\cslhangindent}#1}

\KOMAoption{captions}{tableheading}
\makeatletter
\makeatother
\makeatletter
\@ifpackageloaded{bookmark}{}{\usepackage{bookmark}}
\makeatother
\makeatletter
\@ifpackageloaded{caption}{}{\usepackage{caption}}
\AtBeginDocument{%
\ifdefined\contentsname
  \renewcommand*\contentsname{Table of contents}
\else
  \newcommand\contentsname{Table of contents}
\fi
\ifdefined\listfigurename
  \renewcommand*\listfigurename{List of Figures}
\else
  \newcommand\listfigurename{List of Figures}
\fi
\ifdefined\listtablename
  \renewcommand*\listtablename{List of Tables}
\else
  \newcommand\listtablename{List of Tables}
\fi
\ifdefined\figurename
  \renewcommand*\figurename{Figure}
\else
  \newcommand\figurename{Figure}
\fi
\ifdefined\tablename
  \renewcommand*\tablename{Table}
\else
  \newcommand\tablename{Table}
\fi
}
\@ifpackageloaded{float}{}{\usepackage{float}}
\floatstyle{ruled}
\@ifundefined{c@chapter}{\newfloat{codelisting}{h}{lop}}{\newfloat{codelisting}{h}{lop}[chapter]}
\floatname{codelisting}{Listing}
\newcommand*\listoflistings{\listof{codelisting}{List of Listings}}
\makeatother
\makeatletter
\@ifpackageloaded{caption}{}{\usepackage{caption}}
\@ifpackageloaded{subcaption}{}{\usepackage{subcaption}}
\makeatother
\makeatletter
\@ifpackageloaded{tcolorbox}{}{\usepackage[many]{tcolorbox}}
\makeatother
\makeatletter
\@ifundefined{shadecolor}{\definecolor{shadecolor}{rgb}{.97, .97, .97}}
\makeatother
\makeatletter
\makeatother
\ifLuaTeX
  \usepackage{selnolig}  % disable illegal ligatures
\fi
\IfFileExists{bookmark.sty}{\usepackage{bookmark}}{\usepackage{hyperref}}
\IfFileExists{xurl.sty}{\usepackage{xurl}}{} % add URL line breaks if available
\urlstyle{same} % disable monospaced font for URLs
\hypersetup{
  pdftitle={Oxinfer onboarding},
  pdfauthor={Danielle Newby, Martí Català, Edward Burn},
  colorlinks=true,
  linkcolor={blue},
  filecolor={Maroon},
  citecolor={Blue},
  urlcolor={Blue},
  pdfcreator={LaTeX via pandoc}}

\title{Oxinfer onboarding}
\author{Danielle Newby, Martí Català, Edward Burn}
\date{2024-07-17T00:00:00+01:00}

\begin{document}
\maketitle
\ifdefined\Shaded\renewenvironment{Shaded}{\begin{tcolorbox}[borderline west={3pt}{0pt}{shadecolor}, boxrule=0pt, breakable, enhanced, sharp corners, interior hidden, frame hidden]}{\end{tcolorbox}}\fi

\renewcommand*\contentsname{Table of contents}
{
\hypersetup{linkcolor=}
\setcounter{tocdepth}{2}
\tableofcontents
}
\bookmarksetup{startatroot}

\hypertarget{preface}{%
\chapter*{Preface}\label{preface}}
\addcontentsline{toc}{chapter}{Preface}

\hypertarget{onboarding-to-the-oxinfer-group}{%
\section*{Onboarding to the oxinfer
group}\label{onboarding-to-the-oxinfer-group}}
\addcontentsline{toc}{section}{Onboarding to the oxinfer group}

We've written this book for anyone interested in a working with
databases mapped to the OMOP Common Data Model (CDM) in a tidyverse
inspired approach. That is, human centered, consistent, composable, and
inclusive (see \url{https://design.tidyverse.org/unifying.html} for more
details on these principles).

New to the OMOP CDM? We'd recommend you pare this book with
\href{https://ohdsi.github.io/TheBookOfOhdsi/}{The Book of OHDSI}

New to R? We recommend you compliment the book with
\href{https://r4ds.had.co.nz/}{R for data science}

\hypertarget{citation}{%
\section*{Citation}\label{citation}}
\addcontentsline{toc}{section}{Citation}

TO ADD

\hypertarget{license}{%
\section*{License}\label{license}}
\addcontentsline{toc}{section}{License}

\bookmarksetup{startatroot}

\hypertarget{connect-to-the-database}{%
\chapter{Connect to the database}\label{connect-to-the-database}}

\hypertarget{getting-started}{%
\section{Getting started}\label{getting-started}}

To connect to databases we will use DBI package and CDMConnector, you
can find more information about both packages in their websites:

\begin{itemize}
\item
  DBI package website: \url{https://dbi.r-dbi.org/}
\item
  CDMConnector package website:
  \url{https://darwin-eu.github.io/CDMConnector/}
\end{itemize}

Connect to database (standard way) and set up the environment

To connect to the database, you need to know some parameters of it:
server\_dbi, port, host, server, user, and password.

server\_dbi is different for each one of the databases: e.g.

server\_dbi \textless- ``cdm\_aurum\_202106''

Port is the port used to connect to the database for the moment our
databases use port 5432.

port \textless- ``5432''

Host is the IP of the computer that contains the databases. For the
moment the host of all our current databases is: 163.1.64.2

host \textless- ``163.1.64.2''

Server is the combination of the host and the server\_dbi with a slash
``/'' in between.

server \textless- ``163.1.64.2/cdm\_aurum\_202106''

User and password are provided by the database administrator
(e.g.~Antonella), and they are personal and nontransferable. One of the
first things that we will have to do is to change the password, from the
default one (given by the administrator) to one of our choices (see STEP
2). The username and password are shared by all the databases that we
have on the same port and server, so we need to change it only once.

user \textless- ``\ldots{}''

password \textless- ``\ldots{}''

STEP 0 Install the libraries

The libraries necessary to connect to a database are: DBI, RPostgres,
dplyr, dbplyr, usethis, DatabaseConnector and here. They are all CRAN
libraries so there should not be any problem installing them: if you
have any problem please ask for help. To install them run the following
commands:

STEP 1 Connect to a database for first time

Execute the following commands, make sure that you fill the parameters
of the database accordingly (lines 7-11).

\hypertarget{load-libraries}{%
\section{Load libraries}\label{load-libraries}}

The following libraries will be used in this chapter: DBI, RPostgres,
dplyr, dbplyr, usethis and here. If you do not have them installed you
can install them with the following command:

\begin{Shaded}
\begin{Highlighting}[]
\FunctionTok{install.packages}\NormalTok{(}\FunctionTok{c}\NormalTok{(}\StringTok{"DBI"}\NormalTok{, }\StringTok{"RPostgres"}\NormalTok{, }\StringTok{"dplyr"}\NormalTok{, }\StringTok{"dbplyr"}\NormalTok{, }\StringTok{"usethis"}\NormalTok{, }\StringTok{"here"}\NormalTok{))}
\NormalTok{x }\OtherTok{\textless{}{-}} \DecValTok{1}
\end{Highlighting}
\end{Shaded}

library(``DBI'')

library(``RPostgres'')

\hypertarget{set-connection-details}{%
\section{Set connection details}\label{set-connection-details}}

\hypertarget{substitute-the-next-5-lines-with-the-specifications-of-the-database-that-you}{%
\section{Substitute the next 5 lines with the specifications of the
database that
you}\label{substitute-the-next-5-lines-with-the-specifications-of-the-database-that-you}}

\hypertarget{want-to-connect}{%
\section{want to connect}\label{want-to-connect}}

server\_dbi \textless- ``\ldots{}''

user \textless- ``\ldots{}''

password \textless- ``\ldots{}''

port \textless- ``\ldots{}''

host \textless- ``\ldots{}''

\hypertarget{connect-to-the-database-1}{%
\section{Connect to the database}\label{connect-to-the-database-1}}

db \textless- dbConnect(RPostgres::Postgres(),

\begin{verbatim}
                    dbname = server_dbi, 

port = port, 

   host = host, 
\end{verbatim}

user = user,

\begin{verbatim}
    password = password) 
\end{verbatim}

To check that you are connected run the following commands:

library(``dplyr'')

tbl(db, sql(``SELECT * FROM public.person limit 1''))

If you are connected you should see something like this in your
terminal:

ShapeBackground pattern

Description automatically generated

STEP 2 Change the password (only to be executed the first time a user
connects)

For the new password make sure that you chose a strong password.

After connecting to the database with the temporary password, to change
the password, you must execute the following command in the terminal:

dbGetQuery(db, ``ALTER USER user WITH PASSWORD `new\_password'\,'')

Example:

dbGetQuery(db, ``ALTER USER martics WITH PASSWORD `12345678'\,'')

Disconnect from the database and reconnect with the new password to make
sure that the password change was effective.

dbDisconnect(db)

password \textless- ``new\_password''

db \textless- dbConnect(RPostgres::Postgres(),

\begin{verbatim}
                    dbname = server_dbi, 

 port = port, 
\end{verbatim}

host = host,

\begin{verbatim}
    user = user,

    password = password) 
\end{verbatim}

tbl(db, sql(``SELECT * FROM public.person limit 1''))

If everything worked, you must observe the same output that you observed
before.

STEP 3 Set .Renviron

Including username, password and connection details in an R script is
something that we must avoid at all costs. Otherwise, when we share our
code or upload it in GitHub our account can be vulnerable. So NEVER
include the connection details into an R script!

To avoid this potential problem we are going to store the connection
details of our database in .Renviron file so that we can use them easily
without sharing them.

How .Renviron works? This is a file where we write a list of variables
with their values that we can access from any R script. For example, if
I add to the file:

NAME = ``marti''

When I execute the following command, I will obtain ``marti'' as a
result:

Sys.getenv(``NAME'')

So this two commands would have the same output, but in the second case
without having the same .Renviron file someone external won't be able to
read it.

x \textless- ``marti''

x \textless- Sys.getenv(``NAME'')

To open and set the .Renviron file we have two options:

OPTION 1: open directly the file:

Graphical user interface, application

Description automatically generated

OPTION 2: write the following command in the console:

usethis::edit\_r\_environ()

Once you have it opened you must add the follow to the file:

DB\_USER = ``xxx'' \# Username for the database

DB\_PASSWORD = ``xxx'' \# Password for the database, the new one!!

DB\_PORT = ``5432''

DB\_HOST = ``163.1.64.2''

DB\_SERVER\_name\_database = ``163.1.64.2/xxx'' \# xxx depends on the
database that you

DB\_SERVER\_DBI\_name\_database = ``xxx'' \# want to access, see section
4.

Instead of DB\_SERVER\_DBI\_name\_database we can write a name that we
can remember easily. For example, for CPRD AURUM May 2021 release I
would write in .Renviron file:

DB\_SERVER\_DBI\_aurum202106 = ``cdm\_aurum\_202106''

And call it from the main script as:

server\_dbi \textless- Sys.getenv(``DB\_SERVER\_DBI\_aurum202106'')

But as said we can use any name that we want to use.

Important!! Every time that we change .Renviron file, R must be rebooted
(Session / Restart R, or Ctrl+Shift+F10) so that the changes are
effective.

STEP 4 Check that .Renviron setting works

After rebooting, execute the following code to check if you can connect
to the data base:

library(``DBI'')

library(``RPostgres'')

\hypertarget{use-your-own-.renviron-keys-to-select-the-database-that-you-want-to-access}{%
\section{use your own .Renviron ``keys'' to select the database that you
want to
access}\label{use-your-own-.renviron-keys-to-select-the-database-that-you-want-to-access}}

server\_dbi \textless- Sys.getenv(``DB\_SERVER\_DBI\_name\_database'')

user \textless- Sys.getenv(``DB\_USER'')

password \textless- Sys.getenv(``DB\_PASSWORD'')

port \textless- Sys.getenv(``DB\_PORT'')

host \textless- Sys.getenv(``DB\_HOST'')

db \textless- dbConnect(RPostgres::Postgres(),

\begin{verbatim}
                    dbname = server_dbi, 

    port = port, 

    host = host,

     user = user,

    password = password) 
\end{verbatim}

tbl(db, sql(``SELECT * FROM public.person limit 1''))

If you connected correctly to the database, you must observe the same
output that you observed before.

Connect to the database (OHDSI way)

The connector used in OHDSI packages is different from the one used in
the ``standard way''. To create the connection used in OHDSI packages
for your database you must use the following commands:

library(``DatabaseConnector'')

library(``here'')

server \textless- Sys.getenv(``DB\_SERVER\_name\_database'')

user \textless- Sys.getenv(``DB\_USER'')

password \textless- Sys.getenv(``DB\_PASSWORD'')

port\textless- Sys.getenv(``DB\_PORT'')

host \textless- Sys.getenv(``DB\_HOST'')

connectionDetails \textless-
DatabaseConnector::downloadJdbcDrivers(``postgresql'',

\begin{verbatim}
                                                        here::here()) 
\end{verbatim}

connectionDetails \textless- DatabaseConnector::createConnectionDetails(

\begin{verbatim}
                            dbms = "postgresql", 
\end{verbatim}

server = server,

user = user,

password = password,

port = port,

pathToDriver = here::here())

Important considerations

The SQL dialect of our database is: postgresql. This will be needed in
some functions (manly OHDSI packages) as an input.

targetDialect \textless- ``postgresql''

IMPORTANT!!

The database works with schemas each schema contains different tables
and data. The two schemas that we have access to are:

Public: contains the mapped data and the vocabularies. This schema
should NEVER be used to write.

This should be automatically achieved if people have the correct
privileges and do not use WebAPI privileges.

Results: should be the only schema where the users have read and write
privileges. It contains the cohorts that we create as well as the tables
that we upload.

cdm\_database\_schema \textless- ``public'' vocabulary\_database\_schema
\textless- ``public'' results\_database\_schema\textless- ``results''

Available databases

For the moment we have access to 4 databases:

cdm\_aurum\_202106 - CPRD AURUM 2021\_06 release

p20\_059\_cdm\_aurum - CPRD AURUM for protocol 20\_059.

p20\_000211\_cdm\_aurum - CPRD AURUM for protocol 20\_000211.

cdmgold202007 - CPRD GOLD 2020\_07 release.

A suggestion of data to add to the .Renviron file would be:

DB\_SERVER\_aurum\_202106 = ``163.1.64.2/cdm\_aurum\_202106''

DB\_SERVER\_DBI\_aurum\_202106 = ``cdm\_aurum\_202106''

DB\_SERVER\_aurum\_protocol\_059 = ``163.1.64.2/p20\_059\_cdm\_aurum''

DB\_SERVER\_DBI\_ aurum\_protocol\_059 = ``p20\_059\_cdm\_aurum''

DB\_SERVER\_aurum\_protocol\_211 =
``163.1.64.2/p20\_000211\_cdm\_aurum''

DB\_SERVER\_DBI\_aurum\_protocol\_211 = ``p20\_000211\_cdm\_aurum''

DB\_SERVER\_gold\_202007 = ``163.1.64.2/cdmgold202007''

DB\_SERVER\_DBI\_gold\_202007 = ``cdmgold202007''

\bookmarksetup{startatroot}

\hypertarget{references}{%
\chapter*{References}\label{references}}
\addcontentsline{toc}{chapter}{References}

\hypertarget{refs}{}
\begin{CSLReferences}{0}{0}
\end{CSLReferences}



\end{document}
